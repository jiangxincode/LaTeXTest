\chapter{Process}

Periodic releases of the current draft will take place as the license-drafting
process progresses. Each draft will represent the most current proposed
changes to the GPL. This will take into account all resolved issues, see
4.2, as well as discussions. We plan to release at least two drafts for public
comment. As with all materials and announcements during the discussion
process, these drafts will be available from GPLv3.fsf.org.

\section{Initial Draft Announcement}

The first Discussion Draft of version 3 of the GPL will be released at the
the first International Public Conference, January 16-17, 2006, at the
Massachusetts Institute of Technology, see Appendix A. To accompany the
first discussion draft, we will also release a Rationale Document
explaining the reasons behind each change in an effort to clarify the
nature and necessity of such changes. Similar Rationale documents will
accompany each subsequent Discussion Draft of the license as it is released,
see 2.6.

\section{Publication of Revised Drafts}

At least two discussion drafts of GPL version 3 will be released for public
comment. Publication of the second discussion draft will occur after four
or five months of discussion, issue identification, and resolution. A third
discussion draft may be produced in approximately October 2006, see
Appendix A, after a second or subsequent iterative process of comment,
issue identification, etc. One of these will be the ``last call'' draft,
according to conditions outlined in section 2.4.

\section{Draft Discussion}

All consultation with parties outside the Free Software Foundation and the
Software Freedom Law Center concerning each discussion draft will be a
matter of publicly accessible record available from GPLv3.fsf.org. Written
deliberations from the Discussion Committees will also be available at
GPLv3.fsf.org; sound and video recordings of live events and deliberation
may become available at a later time. We expect to develop this license
through public discussion in a transparent and accessible manner. To that
end every effort will be made to make public all documents pertaining to
the process.

The GPL revision comment process is a matter of information sharing.
Below, in sections 3 and 4.1, we lay out ways that the community as a
whole will tell version 3 drafters of issues with the current license.
They will speak of ways to increase the positive impact of the license on
the world.  The Rationale Documents, outlined in section 2.6, and the
process for Issue Resolution, section 4.2, are designed so that, in turn,
the drafters at FSF can directly address the community and present the
reasoning behind changes.

\section{Last Call Draft}

Either the second or third discussion draft will be designated the
``last call'' draft. This draft will begin a final period of public comment
lasting at least 45 days, ending no later than January 15th, 2007. The
second discussion draft may be designated the last call draft without
further process if there are no major unresolved issues after full
discussion of the initial draft.

\section{Promulgation}

No later than March 2007, and preferably on January 15, 2007, at the
conclusion of the last call process, with all issues resolved, the Free
Software Foundation will formally adopt version 3 of the GNU General Public
License. At that time, the Free Software Foundation will relicense under
GPL version 3 or later all parts of the GNU Project for which the Free
Software Foundation is the copyright holder. All parties with authority to
relicense programs whose current license terms are ``GPL version 2 only''
will then be in a position to decide whether to relicense their code. The
Free Software Foundation hopes that Discussion Committee members will
encourage the relicensing of such works, which is at the discretion of the
relevant copyright holders.

\section{Rationales}

To make the commentary process easier and to keep the license-drafting
process open, each successive draft of the GPL version 3 will be
accompanied by a Rationale Document. This document will explain all planned
changes in light of the purposes of the license and the freedoms it protects.
It will also summarize the public commentary and response relevant
to any changed portions of the license. In this, the Rationale Documents
will complement the opinion papers issued by the Free Software Foundation
regarding resolution of individual issues as identified by the Discussion
Committees, see section 4.2. Rationale Documents will be available
through the website GPLv3.fsf.org.

\section{Outreach}

Transparency does not guarantee widespread distribution; we need to
work for that. Much of our effort will therefore be invested in publication
and outreach. All information submitted by the public through the revision
process will be passed on to the drafters, whether by direct comment
submission, Discussion Committee analysis, or transcript of International
Conference meetings, and all of it will remain available to the public at
GPLv3.fsf.org.

Community members who share their experiences with the drafters are
encouraged to share them with the rest of the community. People with
knowledge of the GPL and the free software movement can educate their
fellow community members, as well as people with no previous knowledge of
free software. In an effort to extend the process to the greatest possible
number of settings, a team of editors from the FSF will help develop
comprehensive issue guides and introductions.

The process of revising the GPL is an opportunity for the community
that cares about freedom to educate the rest of the societies they live in.
Everyone concerned with the GPL will be asked to examine the license
in detail and articulate its impact and possible ways for it to better protect
their and others' freedoms. For successful drafting and spread of version
3 of the GPL, this commentary must not only educate the drafters but also
the community and public at large.
