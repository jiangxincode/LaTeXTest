\chapter{Issue Management and Resolution}


From the Foundation's point of view, the revision process is characterized
by the presentation and closure of issues revealed in draft discussions.

\section{Forming Issues}

The purpose of this public discussion process is to encourage information
about the GPL and its role in the expansion and protection of software
freedom. This purpose is empty without public commentary. In order to
make the most well-informed changes possible to the GPL, we seek commentary
from a wide selection of the public. Comments and suggestions
are encouraged at \url{GPLv3.fsf.org} as well as in person at any of our
International Conferences, see Appendix A.

After someone has made a comment, either directly to \url{GPLv3.fsf.org}
or in a discussion at an International Conference, a number of steps will be
taken to associate that comment with one or more currently known issues.
While comments are the substance of the feedback process, issues are
the containers through which they will move.

If the comment or suggestion presents a problem not already identified
as an issue, it will be forwarded to the appropriate Discussion Committee
where it will join other comments in the identification of a new issue.

For each comment to \url{GPLv3.fsf.org}, this process has three steps.
First, when making the comment the commentator can specify what portion
of the license or issue about the license their comment addresses.
Once submitted, the comment will be read by an associate member of the
Free Software Foundation who will direct it to the appropriate Discussion
Committee either for issue identification, if no preexisting issues matches
with the comment, or to inform the discussion of the particular issue it
addresses.

After making a comment at \url{GPLv3.fsf.org}, the person involved will
be given a comment-identifying number that he or she can use to see
towards what issue and Discussion Committee the comment was directed,
as well as other comments on the issue and the documents relevant to
its discussion (transcripts; Discussion Committee analysis, see 3; Draft
Rationale documents, see 2.6; FSF Opinion documents, see 4.2, etc.).

\section{Issue Resolution}

Each issue identified in the course of public participation can be resolved
in one of four ways: by modification of the license draft, by alteration of
descriptive material, by advice concerning the use of the license, or an
issue may not require any change. Discussion Committees will characterize
issues as Major or Minor. Major issues will be placed on the agenda of
all other Discussion Committees and, until resolved, may be placed on the
agenda for successive International meetings. All issues unresolved at the
end of each drafting stage will be carried over for discussion and
resolution during the next discussion stage. All issues not resolved before
the issuance of the last discussion draft will be finally determined by the
Free Software Foundation at the close of the last call period. All Major
issues resolved by the Foundation will be described by a written opinion,
publicly available, at \url{GPLv3.fsf.org}.
