\chapter*{Introduction}

\addcontentsline{toc}{chapter}{Introduction}


The technologies of Software development and deployment have changed
dramatically since 1991 while the GNU General Public License (``the GPL'')
has remained unmodified, at version level 2. This is extraordinary longevity
for any widely-employed legal instrument. The durability of the GPL is even
more surprising when one takes into account the differences between the
free software community at the time of version 2's release and the situation
prevailing in 2005.

Today, the GPL is employed by tens of thousands of software
projects around the world and while the Free Software Foundation's body
of GPL licensed works is vital, it consists of no more than a tiny fraction
of them. GPL'd software runs on or is embedded in devices ranging from
cellphones, PDAs, and home networking appliances to mainframes and
supercomputing clusters. Independent software developers around the
world, as well as every large corporate IT buyer and seller, and a
surprisingly large number of individuals, interact with the GPL. Moreover,
free software transcends national boundaries. The GPL's use is global.

Richard M. Stallman, who founded the free software movement and
who was the author of the GNU GPL, released version 2 in 1991 after
taking legal advice and collecting developer's opinions concerning version
1 of the license, which had been in use since 1989. Given that the Free
Software Foundation directly controlled the licensing of the GNU project,
which comprised the largest then-existing collection of copylefted software
assets, no public comment process and no significant interim transition
period seemed necessary. The Free Software Foundation immediately relicensed
the components of the GNU Project and in Finland Linus Torvalds
adopted GPL Version 2 for his operating system kernel, called Linux.

Many provisions of the GPL could benefit from modification to fit today's
circumstances and to reflect what we have learned from experience with
version 2. Given the scale of revision it seems proper to approach the work
through public discussion in a transparent and accessible manner.

The Free Software Foundation plans to decide the contents of version
3 of the GPL through the fullest possible discussion with the most diverse
possible community of drafters and users. A major goal is to identify every
issue affecting every user, and to resolve those issues.

For these reasons, the process of GPL revision will be a time of
self-examination. Consequently, the process of drafting and adopting changes
must be as close to “best practices” as possible, for both lawyers and lay
people. Experience has thrown new light on the text of the current GPL.
The utility of some provisions has altered over time, while others need
to increase their reach in order to protect freedom in the new world of
software. Most of the issues caused by this gradual development of the
software world can be addressed with minor changes in the text of the
GPL.

People who use software, whether they receive copies on CD, or interact
with remote installations of the software, have the right to share and
improve that software. (Clearly, many, perhaps most, will not modify
software; but they share it and desire fixes and improvements. This means
they and others must have the right.)

While the GPL is the most popular Free Software License, followed by
the LGPL, a significant set of free software is licensed under other terms
which are not compatible with version 2 of the GPL. Version 3 of the GPL
will provide compatibility with more non-GPL free licenses.

Our primary concern remains, as it has been from the beginning, to
give users freedom that they can rely on. As the community around free
software has grown larger the issues involved in this creation and
protection of freedom have grown more diverse and complex. Therefore,
we have consulted, formally and informally, a very broad array of
participants in the free software community, from industry, the academy,
and the garage. Those conversations have occurred in many countries and
several languages, over almost two decades, as the technology of software
development and distribution changed around us. We recognize that the
best protection of freedom is a growing and vital community of the free and
we hope the spread of knowledge inherent in public discussion of version
3 of the GPL drafts will continue to support and nurture this community.

When a discussion draft of version 3 of the GPL is released, the pace
of the revision conversation will change, as a particular proposal becomes
the centerpiece. The reversioning of the GPL is a crucial moment in the
evolution of the free software community; and the Foundation intends to
meet its responsibilities to the makers, distributors, and users of free
software. In doing so, we hope to hear all relevant points of view, and to
make decisions that fit the many circumstances that arise in the use and
development of GPL-covered software.