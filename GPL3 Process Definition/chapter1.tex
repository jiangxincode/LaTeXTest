\chapter{Objectives}


In drafting a new version of the GPL, the Free Software Foundation have
been guided by a few basic principles. These will inform the processes
of discussing and promulgating the license as described herein. These
principles and their impact on the discussion process are listed below.

\section{A Global License}

As a legal document, the GPL licenses copyrighted material for
modification and redistribution in every one of the world's systems of
copyright law.
Ours is an approach that most legal drafters would do anything possible
to avoid. Publishers in general do not use worldwide copyright licenses:
they try to tailor their licensing arrangements to local legal requirements
for each system in which their works are distributed. Publishers rarely
license the redistribution of modified or derivative works. When they do,
those licenses are tailored to the specific setting.

But free software requires legal arrangements that permit copyrighted
works to follow arbitrary trajectories, in both geographic and genetic terms.
Indeed, modified versions of free software works are distributed from hand
to hand across borders in a pattern that no copyright holder could or should
be permitted to trace.

GPL version 2 performed the task of globalization relatively well, because
its design was elegantly limited to a minimum set of copyright requirements.
Every signatory to the Berne Convention---which means most
countries in the world---must offer those principles in their national
legislation, in one form or another. But GPL version 2 was constructed only
with attention to the details of US law. To the extent possible, without any
fundamental changes, version 3 of the GPL should reduce the difficulties
of internationalization. Version 3 should more fully approximate the
otherwise unsought ideal of the global copyright license.

\section{Protection of Existing Freedoms}

Our cardinal principle is to make no change impeding any of the four basic
freedoms for software users that the free software movement enshrined in
GPL version 2: to run, study, copy, modify and redistribute software. (It
goes without saying that people have the freedom to run a program under
the GPL.) These freedoms are as important in version 3 of the GPL as they
were in version 2. Honoring the commitment stated in earlier versions of
the license, we will preserve these basic rights.

We have judged all changes proposed since the adoption of GPL version 2
against those yardsticks and we will present, in the rationale documents
described in section 2.6, reasons tying our changes to those fundamental
freedoms. Parties who question changes should recognize when
writing their comments that these freedoms remain the cornerstone of the
license. We will evaluate all proposed changes with reference to them.

\section{Do No Harm}

Unintended consequences can imperil freedom. In approaching GPL version 3
we recognize the enormous expansion in the use of free software
since 1991, as well as the many modes of use and distribution that have
been invented since. These make the risks of unintended consequences
much more severe than when the GPL was last modified.

A large part of the value of the public discussion and issue development
described in this document is the identification of unintended consequences
worldwide. This is vital to ensuring that version 3 of the GPL is a
global license that works as intended in all major legal systems.

Our revision process is intended to make an exhaustive analysis of
each considered change in order to explore as much as possible, in as
many situations as possible, with as many users and distributors as possible.

\section{Consulting the Community}

In short, the essence of the drafting process here described is to make
it possible for the Free Software Foundation to decide the contents of the
GPL through the fullest possible discussion with the most diverse possible
community of drafters and users. Ideally, we would identify every issue
affecting every user of the license and resolve these issues with a full
consideration of their risks and benefits. In order to accomplish such a
large task, the discussion process involves individual community members
and Discussion Committees that represent different types of users and
distributors.

Each proposed change and the resolution of each issue needs the
fullest description of risks and benefits, as laid out in section 4.1.

The Discussion Committees, as described in section 3, will serve as
important centralized points among the different types of user. Among
other actions, their role will be to identify issues from the large body
of user experience and develop those issues for full presentation and
resolution by the Free Software Foundation.
