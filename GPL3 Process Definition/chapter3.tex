\chapter{Committees}


Dealing with what will probably be extensive public comments is the task
of the Discussion Committees, which must structure the flow of comments
into issues that can be productively analyzed and whose proposed solutions
can be debated. Their work in discovering, developing, and presenting
issues is the heart of the version 3 public discussion process.

\section{Composition}

Most issues for GPLv3 are global. Therefore we plan to form committees
including all categories of relationship to the GPL itself, rather than
adopting a regional formation. Thus, elements of the larger GPL community
around which Discussion Committees will be formed will include large
and small enterprises, both public and private; vendors, commercial and
noncommercial redistributors; development projects that use the GPL as
a license for their programs; development projects that use other free
software licenses, but are invested in the contents of the GPL; and
unaffiliated individual developers and people who use software.

Coincident with the publication of this document, the Free Software
Foundation will issue invitations to participate in Discussion Committees.
These invitations will form nuclei of people. We hope that our invitations
will result in Committees that reflect the full breadth of opinion within
those sections of the community they functionally represent. But we
expect that the Committees themselves will choose to invite additional
participants---people whose commitment to the license is undoubted---to
add the weights of their opinions to the deliberations. Such invitations,
issued after the invitation of the process, shall be by majority vote of
each Committee as already constituted.

\section{Process Commitments}

The Committees and their chairs should actively encourage public
participation from the sectors of the public they represent.

In addition, Committees are responsible for developing all the opinions
and analysis concerning issues they identify from the stream of
commentary. As each Committee feels that an issue has been fully discussed
among its members, it will be expected to present to the Free Software
Foundation its deliberation and analysis of the issue as well as a summary
of the public comment that informed its position. Where technically
feasible, both the deliberations of the Discussion Committees and the
arguments and analysis that they present to the Free Software Foundation
will be published at \url{GPLv3.fsf.org}.

At the conclusion of the public discussion process, we hope to ask
members of the Discussion Committees to assist the Free Software Foundation
in promulgating the new license; that is, to work with the knowledge
gained from their central position within the discussion and revision
process to advocate the relicensing of existing GPL programs under version
3 of the GPL.

\section{Organizational Structure}

Discussion Committees should be free to choose their own working structure.
The Free Software Foundation will provide a template working structure
for each committee.

Discussion Committees should operate largely through network-based
communication, voice and data, synchronously and asynchronously. They
will organize themselves through regular meetings and web-based
interactions, encourage public comment and participation, identify and
discuss issues, and present those issues and all relevant argument to the
Free Software Foundation for ultimate resolution.
