\chapter{Other Concerns}


\section{LGPL}

The Free Software Foundation may present drafts of LGPL along with
drafts of GPL subsequent to the first discussion draft of GPLv3. Such
drafts would also be subject to public comment and issue resolution.

\section{Support of the Revision Process}

The revision process is financially supported by donors to the Free
Software Foundation, the Free Software Foundation's associate members, and
a grant from Stichting NLnet. Logistical, legal, and technical support for
this process is also provided by the Software Freedom Law Center, acting
as the Free Software Foundation's outside counsel. The SFLC is supported
by a variety of donors including vendors and those who use free software.

Aside from resources contributed by the Free Software Foundation and
the Software Freedom Law Center, this process will be supported, only
to the extent of logistical provision for International Meetings, by industry
organizations hosting the events. Outside logistical support is accepted
only in order to ensure that participants around the world will have the
maximum possible level of access to the discussion process of version
3 of the GPL. All participants in the discussion process can therefore be
assured of equal treatment for their interests and concerns.

\section{Public Statements}

During this process the Free Software Foundation will make public
statements concerning the process, deadlines, issues, comments, and drafts.
Such public statements will be made through announcements at \url{GPLv3.fsf.org},
and by messages to mailing lists to which parties can subscribe. The Free
Software Foundation and SFLC will not hold confidential communications with
others concerning version 3 of the GPL. Public commentary on these
announcements, as with all comments relating to the GPL version 3 discussion
process, should be routed through the GPL comment system described in
section 4.1. Interested members of the Press
should see 5.4 below.

\section{Press Contact}

Press contacts may occur and statements may be issued to the press
through the Free Software Foundation and the Software Freedom Law
Center. All such statements will be published at \url{GPLv3.fsf.org} or
referenced there. Press interested in covering this process should
follow the contact information available at the website or write to
\href{maito:press@GPLv3.fsf.org}{press@GPLv3.fsf.org}.
