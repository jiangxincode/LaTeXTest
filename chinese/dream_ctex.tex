% class :             ctexbook, ctexart, ctexrep....
% package supports :  none
% coding :            UTF-8
% compile command :   xelatex
% chinese fonts :     use system TTF (fc-list)

\documentclass[nofonts]{ctexbook}
\usepackage{indentfirst}                % 每章、节首段缩进
\usepackage[CJKbookmarks,               % 中文书签
            colorlinks=true,            % 允许彩色链接
            linkcolor=magenta]{hyperref}% 超链接包

\setCJKfamilyfont{caiyun}{STCaiyun}
\setCJKfamilyfont{songti}{AR PL SungtiL GB}
\setCJKfamilyfont{fangsong}{STFangsong}
\setCJKfamilyfont{heiti}{WenQuanYi Zen Hei}
\setCJKfamilyfont{kaiti}{AR PL UKai CN}
\setCJKfamilyfont{xinwei}{STXinwei}
\setCJKfamilyfont{lishu}{LiSu}
\setCJKmainfont[BoldFont={WenQuanYi Zen Hei}, ItalicFont={STFangsong}]{AR PL SungtiL GB}
\setCJKsansfont{WenQuanYi Zen Hei}
\setCJKmonofont{AR PL UKai CN}

\newcommand*{\song}{\CJKfamily{songti}}   % 宋体
\newcommand*{\fs}{\CJKfamily{fangsong}}   % 仿宋
\newcommand*{\hei}{\CJKfamily{heiti}}     % 黑体
\newcommand*{\kai}{\CJKfamily{kaiti}}     % 楷书
\newcommand*{\wei}{\CJKfamily{xinwei}}    % 新魏
\newcommand*{\lishu}{\CJKfamily{lishu}}   % 隶书
\newcommand*{\cy}{\CJKfamily{caiyun}}     % 彩云

\setlength{\parindent}{2em}             % 缩进2个字母M的宽度

\CTEXsetup[name={第,回}]{chapter}
\CTEXsetup[format={\raggedright}]{chapter}
\CTEXsetup[nameformat={\LARGE \cy}]{chapter}
\CTEXsetup[titleformat=\kai ]{chapter}
\CTEXsetup[aftername={\vskip 3ex }]{chapter}
\CTEXsetup[format+={\zihao{4}}]{chapter}

\begin{document}
\title{\lishu \Huge 紅\\ 樓\\ 夢\ \\ \ \\}
\author{\framebox{曹雪芹}}
\date{猴年马月}

\setlength\baselineskip{4.5ex}

\pagecolor{black} \color{white}
\maketitle
\pagecolor{white} \color{black}

\tableofcontents

\chapter{甄士隐梦幻识通灵\ \ 贾雨村风尘怀闺秀}

    此开卷第一回也。作者自云:因曾历过一番梦幻之后,故将真事隐去,而借``通灵''
之说,撰此$<<$石头记$>>$一书也。故曰``甄士隐''云云。但书中所记何事何人?
自又云:``今风尘碌碌, 一事无成,忽念及当日所有之女子,一一细考较去,觉其行止
见识,皆出于我之上。何我堂堂须眉,诚不若彼裙钗哉?实愧则有余,悔又无益之大无可
如何之日也! 当此,则自欲将已往所赖天恩祖德,锦衣纨袴之时,饫甘餍肥之日,背父兄
教育之恩,负师友规谈之德, 以至今日一技无成,半生潦倒之罪,编述一集, 以告天
下人:我之罪固不免, 然闺阁中本自历历有人,万不可因我之不肖,自护己短, 一并使
其泯灭也。虽今日之茅椽蓬牖, 瓦灶绳床,其晨夕风露,阶柳庭花,亦未有妨我之襟怀
笔墨者。虽我未学, 下笔无文,又何妨用假语村言,敷演出一段故事来,亦可使 闺阁
昭传,复可悦世之目,破人愁闷,不亦宜乎?''故曰``贾雨村''云云。
\marginpar{\small\kai\color{red} 此处可以有掌声}

%测试粗体、斜体
\textbf{
    此回中凡用``梦''用``幻''等字,是提醒阅者眼目,亦是此书立意本旨。}

\textit{
    列位看官:你道此书从何而来?说起根由虽近荒唐,细按则深有趣味。待在下将此
来历注明,方使阅者了然不惑。}
\marginpar{\small\kai\color{red} 注意这三段字体的变化。粗体、斜体是setCJKmainfont中独立定义的不同字体}

    原来女娲氏炼石补天之时,于大荒山无稽崖练成高经十二丈,方经二十四丈顽石
三万六千五百零一块。娲皇氏只用了三万六千五百块,只单单剩了一块未用,便弃在
此山青埂峰下。谁知此石自经煅炼之后,灵性已通,因见众石俱得补天,独自己无材不
堪入选,遂自怨自叹,日夜悲号惭愧。


    一日, 正当嗟悼之际,俄见一僧一道远远而来,生得骨格不凡,丰神迥异,说说笑
笑来至峰下, 坐于石边高谈快论。先是说些云山雾海神仙玄幻之事,后便说到红尘中
荣华富贵。此石听了,不觉打动凡心,也想要到人间去享一享这荣华富贵,但自恨粗蠢
,不得已,便口吐人言,向那僧道说道:``大师,弟子蠢物,不能见礼了。适闻二位谈那人
世间荣耀繁华,心切慕之。弟子质虽粗蠢,性却稍通,况见二师仙形道体,定非凡品,必
有补天济世之材,利物济人之德。如蒙发一点慈心,携带弟子得入红尘,在那富贵场中,
温柔乡里受享几年,自当永佩洪恩,万劫不忘也。''二仙师听毕,齐憨笑道:``善哉,善
哉!那红尘中有却有些乐事,但不能永远依恃,况又有`美中不足,好事多魔'
\marginpar{\small\kai 现通作``好事多磨''}八个字紧
相连属,瞬息间则又乐极悲生,人非物换,究竟是到头一梦,万境归空,倒不如不去的
好。''这石凡心已炽,那里听得进这话去,乃复苦求再四。二仙知不可强制,乃叹道:``此
亦静极怂级*, \footnote{此处未查原文, 疑为生僻字}
无中生有之数也。既如此,我们便携你去受享受享,只是到不得意时,切
莫后悔。''石道:``自然,自然。''那僧又道:``若说你性灵,却又如此质蠢,并更无奇贵之
处。如此也只好踮脚而已。也罢,我如今大施佛法助你助,待劫终之日,复还本质,以了
此案。你道好否?''石头听了,感谢不尽。那僧便念咒书符,大展幻术,将一块大石登时
变成一块鲜明莹洁的美玉,且又缩成扇坠大小的可佩可拿。那僧托于掌上,笑道:``形
体倒也是个宝物了!还只没有,实在的好处,须得再镌上数字,使人一见便知是奇物方
妙。然后携你到那昌明隆盛之邦,诗礼簪缨之族,花柳繁华地,温柔富贵乡去安身乐业
。''石头听了,喜不能禁,乃问:``不知赐了弟子那几件奇处,又不知携了弟子到何地方?
望乞明示,使弟子不惑。''那僧笑道:``你且莫问,日后自然明白的。''说着,便袖了这石,
同那道人飘然而去,竟不知投奔何方何舍。	

    后来,又不知过了几世几劫,因有个空空道人访道求仙,忽从这大荒山无稽崖青
埂峰下经过,忽见一大块石上字迹分明,编述历历。空空道人乃从头一看,原来就是无
材补天,幻形入世,蒙茫茫大士,渺渺真人携入红尘,历尽离合悲欢炎凉世态的一段故
事。后面又有一首偈云:

\begin{verse} \wei
    无材可去补苍天,枉入红尘若许年。\\
    此系身前身后事,倩谁记去作奇传?
\end{verse}

    诗后便是此石坠落之乡,投胎之处,亲自经历的一段陈迹故事。其中家庭闺阁琐事,
以及闲情诗词倒还全备,或可适趣解闷,然朝代年纪,地舆邦国,却反失落无考。

    空空道人遂向石头说道: ``石兄,你这一段故事,据你自己说有些趣味,故编写在
此,意欲问世传奇。据我看来,第一件,无朝代年纪可考,第二件,并无大贤大忠理朝廷
治风俗的善政, 其中只不过几个异样女子,或情或痴,或小才微善,亦无班姑,蔡女之
德能。我纵抄去,恐世人不爱看呢。''石头笑答道:``我师何太痴耶!若云无朝代可考,今
我师竟假借汉唐等年纪添缀, 又有何难?但我想,历来野史,皆蹈一辙,莫如我这不借
此套者, 反倒新奇别致,不过只取其事体情理罢了,又何必拘拘于朝代年纪哉!再者,
市井俗人喜看理治之书者甚少,爱适趣闲文者特多。历来野史,或讪谤君相,或贬人妻
女, 奸淫凶恶,不可胜数。更有一种风月笔墨,其淫秽污臭,屠毒笔墨,坏人子弟,又不
可胜数。至若佳人才子等书,则又千部共出一套,且其中终不能不涉于淫滥,以致满纸
潘安, 子建,西子,文君,不过作者要写出自己的那两首情诗艳赋来,故假拟出男女二
人名姓,又必旁出一小人其间拨乱,亦如剧中之小丑然。且鬟婢开口即者也之乎,非文
即理。故逐一看去,悉皆自相矛盾,大不近情理之话,竟不如我半世亲睹亲闻的这几个
女子,虽不敢说强似前代书中所有之人,但事迹原委,亦可以消愁破闷,也有几首歪诗
熟话,可以喷饭供酒。至若离合悲欢,兴衰际遇,则又追踪蹑迹,不敢稍加穿凿,徒为供
人之目而反失其真传者。今之人,贫者日为衣食所累,富者又怀不足之心,纵然一时稍
闲, 又有贪淫恋色,好货寻愁之事,那里去有工夫看那理治之书?所以我这一段故事,
也不愿世人称奇道妙, 也不定要世人喜悦检读,只愿他们当那醉淫饱卧之时,或避世
去愁之际,把此一玩,岂不省了些寿命筋力?就比那谋虚逐妄,却也省了口舌是非之害
,腿脚奔忙之苦。再者,亦令世人换新眼目,不比那些胡牵乱扯,忽离忽遇,满纸才人淑
女,子建文君红娘小玉等通共熟套之旧稿。我师意为何如?''
									

    空空道人听如此说, 思忖半晌,将$<<$石头记$>>$再检阅一遍,因见上面虽有些指奸
责佞贬恶诛邪之语,亦非伤时骂世之旨,及至君仁臣良父慈子孝,凡伦常所关之处,皆
是称功颂德, 眷眷无穷,实非别书之可比。虽其中大旨谈情,亦不过实录其事,又非假
拟妄称, 一味淫邀艳约,私订偷盟之可比。因毫不干涉时世,方从头至尾抄录回来,问
世传奇。从此空空道人因空见色,由色生情,传情入色,自色悟空,遂易名为情僧,改
$<<$石头记$>>$为《情僧录》。东鲁孔梅溪则题曰$<<$风月宝鉴$>>$。后因曹雪芹
于悼红轩中披阅十载,增删五次,纂成目录,分出章回,则题曰$<<$金陵十二钗$>>$。并题
一绝云:

\begin{verse} \wei
    满纸荒唐言,一把辛酸泪!\\
    都云作者痴,谁解其中味?
\end{verse}

    出则既明,且看石上是何故事。按那石上书云:

\begin{quotation} \fs
    当日地陷东南, 这东南一隅有处曰姑苏,有城曰阊门者,最是红尘中一二等富贵
风流之地。 这阊门外有个十里街,街内有个仁清巷,巷内有个古庙,因地方窄狭,人皆
呼作葫芦庙。庙旁住着一家乡宦,姓甄,名费,字士隐。嫡妻封氏,情性贤淑,深明礼义。
家中虽不甚富贵,然本地便也推他为望族了。因这甄士隐禀性恬淡,不以功名为念,每
日只以观花修竹, 酌酒吟诗为乐,倒是神仙一流人品。只是一件不足:如今年已半百,
膝下无儿,只有一女,乳名唤作英莲,年方三岁。


    一日,炎夏永昼,士隐于书房闲坐,至手倦抛书,伏几少憩,不觉朦胧睡去。梦至一
处,不辨是何地方。忽见那厢来了一僧一道,且行且谈。只听道人问道:``你携了这蠢物
,意欲何往?''那僧笑道:``你放心,如今现有一段风流公案正该了结,这一干风流冤家,
尚未投胎入世。 趁此机会,就将此蠢物夹带于中,使他去经历经历。''那道人道:``原来
近日风流冤孽又将造劫历世去不成? 但不知落于何方何处?''那僧笑道:``此事说来好
笑,竟是千古未闻的罕事。只因西方灵河岸上三生石畔,有绛珠草一株,时有赤瑕宫神
瑛侍者, 日以甘露灌溉,这绛珠草始得久延岁月。后来既受天地精华,复得雨露滋养,
遂得脱却草胎木质,得换人形,仅修成个女体,终日游于离恨天外,饥则食蜜青果为膳,
渴则饮灌愁海水为汤。只因尚未酬报灌溉之德,故其五内便郁结着一段缠绵不尽之
意。恰近日这神瑛侍者凡心偶炽,乘此昌明太平朝世,意欲下凡造历幻缘,已在警幻仙
子案前挂了号。警幻亦曾问及,灌溉之情未偿,趁此倒可了结的。那绛珠仙子道:`他是
甘露之惠,我并无此水可还。他既下世为人,我也去下世为人,但把我一生所有的眼泪
还他, 也偿还得过他了。'因此一事,就勾出多少风流冤家来,陪他们去了结此案。''那
道人道: ``果是罕闻。实未闻有还泪之说。想来这一段故事,比历来风月事故更加琐碎
细腻了。''那僧道:``历来几个风流人物,不过传其大概以及诗词篇章而已,至家庭闺阁
中一饮一食,总未述记。再者,大半风月故事,不过偷香窃玉,暗约私奔而已,并不曾将
儿女之真情发泄一二。想这一干人入世,其情痴色鬼,贤愚不肖者,悉与前人传述不同
矣。''那道人道:``趁此何不你我也去下世度脱几个,岂不是一场功德?''那僧道:``正合
吾意,你且同我到警幻仙子宫中,将蠢物交割清楚,待这一干风流孽鬼下世已完,你我
再去。如今虽已有一半落尘,然犹未全集。''道人道:``既如此,便随你去来。''
\end{quotation}

    却说甄士隐俱听得明白,但不知所云``蠢物''系何东西。遂不禁上前施礼,笑问道:
``二仙师请了。''那僧道也忙答礼相问。士隐因说道:``适闻仙师所谈因果,实人世罕闻
者。但弟子愚浊,不能洞悉明白,若蒙大开痴顽,备细一闻,弟子则洗耳谛听,稍能警省,
亦可免沉伦之苦。''二仙笑道:``此乃玄机不可预泄者。到那时不要忘我二人,便可跳
出火坑矣。 ''士隐听了,不便再问。因笑道:``玄机不可预泄,但适云`蠢物',不知为何,
或可一见否? ''那僧道:``若问此物,倒有一面之缘。''说着,取出递与士隐。士隐接了看
时,原来是块鲜明美玉,上面字迹分明,镌着``通灵宝玉''四字,后面还有几行小字。正
欲细看时,那僧便说已到幻境,便强从手中夺了去,与道人竟过一大石牌坊,上书四个
大字,乃是``太虚幻境''。两边又有一幅对联,道是:


\begin{verse} \wei
    假作真时真亦假,无为有处有还无。
\end{verse}

    士隐意欲也跟了过去,方举步时,忽听一声霹雳,有若山崩地陷。士隐大叫一声,
定睛一看,只见烈日炎炎,芭蕉冉冉,所梦之事便忘了大半。又见奶母正抱了英莲走来。
士隐见女儿越发生得粉妆玉琢,乖觉可喜,便伸手接来,抱在怀内,斗他顽耍一回,又带
至街前,看那过会的热闹。方欲进来时,只见从那边来了一僧一道:那僧则癞头跣脚,
那道则跛足蓬头,疯疯癫癫,挥霍谈笑而至。及至到了他门前,看见士隐抱着英莲,那僧
便大哭起来,又向士隐道:``施主,你把这有命无运,累及爹娘之物,抱在怀内作甚?''士隐
听了,知是疯话,也不去睬他。那僧还说:``舍我罢,舍我罢!''士隐不耐烦,便抱女儿撤身
要进去,那僧乃指着他大笑,口内念了四句言词道:


\begin{verse} \wei
    惯养娇生笑你痴,菱花空对雪澌澌。\\
    好防佳节元宵后, 便是烟消火灭时。
\end{verse}

    士隐听得明白,心下犹豫,意欲问他们来历。只听道人说道: ``你我不必同行,就此
分手,各干营生去罢。三劫后,我在北邙山等你,会齐了同往太虚幻境销号。''那僧道:
``最妙,最妙!''说毕,二人一去,再不见个踪影了。士隐心中此时自忖:这两个人必有来
历,该试一问,如今悔却晚也。


    这士隐正痴想, 忽见隔壁葫芦庙内寄居的一个穷儒---姓贾名化,表字时飞,别
号雨村者走了出来。这贾雨村原系胡州人氏,也是诗书仕宦之族,因他生于末世,父母
祖宗根基已尽, 人口衰丧,只剩得他一身一口,在家乡无益,因进京求取功名,再整基
业。自前岁来此,又淹蹇住了,暂寄庙中安身,每日卖字作文为生,故士隐常与他交接。
当下雨村见了士隐,忙施礼陪笑道:``老先生倚门伫望,敢是街市上有甚新闻否?''士隐
笑道: ``非也。适因小女啼哭,引他出来作耍,正是无聊之甚,兄来得正妙,请入小斋一
谈,彼此皆可消此永昼。''说着,便令人送女儿进去,自与雨村携手来至书房中。小童献
茶。方谈得三五句话,忽家人飞报:``严老爷来拜。''士隐慌的忙起身谢罪道:``恕诳驾之
罪,略坐,弟即来陪。''雨村忙起身亦让道:``老先生请便。晚生乃常造之客,稍候何妨。''
说着,士隐已出前厅去了。


    这里雨村且翻弄书籍解闷。忽听得窗外有女子嗽声,雨村遂起身往窗外一看,原
来是一个丫鬟, 在那里撷花,生得仪容不俗,眉目清明,虽无十分姿色,却亦有动人之
处。雨村不觉看的呆了。那甄家丫鬟撷了花,方欲走时,猛抬头见窗内有人,敝巾旧服,
虽是贫窘,然生得腰圆背厚,面阔口方,更兼剑眉星眼,直鼻权腮。这丫鬟忙转身回避,
心下乃想:``这人生的这样雄壮,却又这样褴褛,想他定是我家主人常说的什么贾雨村
了, 每有意帮助周济,只是没甚机会。我家并无这样贫窘亲友,想定是此人无疑了。怪
道又说他必非久困之人。 ''如此想来,不免又回头两次。雨村见他回了头,便自为这女
子心中有意于他,便狂喜不尽,自为此女子必是个巨眼英雄,风尘中之知己也。一时小
童进来, 雨村打听得前面留饭,不可久待,遂从夹道中自便出门去了。士隐待客既散,
知雨村自便,也不去再邀。

    一日,早又中秋佳节。士隐家宴已毕,乃又另具一席于书房,却自己步月至庙中来
邀雨村。 原来雨村自那日见了甄家之婢曾回顾他两次,自为是个知己,便时刻放在心
上。今又正值中秋,不免对月有怀,因而口占五言一律云:

\begin{verse} \wei
    未卜三生愿,频添一段愁。\\
    闷来时敛额,行去几回头。\\
    自顾风前影,谁堪月下俦?\\
    蟾光如有意,先上玉人楼。
\end{verse}

    雨村吟罢,因又思及平生抱负,苦未逢时,乃又搔首对天长叹,复高吟一联曰:

\begin{verse} \wei
    玉在椟中求善价,钗于奁内待时飞。
\end{verse}

    恰值士隐走来听见,笑道:``雨村兄真抱负不浅也!''雨村忙笑道:``不过偶吟前人
之句,何敢狂诞至此。''因问:``老先生何兴至此?''士隐笑道:``今夜中秋,俗谓`团圆之节',
想尊兄旅寄僧房,不无寂寥之感,故特具小酌,邀兄到敝斋一饮,不知可纳芹意否?''雨村
听了,并不推辞,便笑道:``既蒙厚爱,何敢拂此盛情。''说着,便同士隐复过这边书院中来。
须臾茶毕,早已设下杯盘,那美酒佳肴自不必说。 二人归坐,先是款斟漫饮,次渐谈至
兴浓,不觉飞觥限 起来。当时街坊上家家箫管,户户弦歌,当头一轮明月,飞彩凝辉,
二人愈添豪兴,酒到杯干。雨村此时已有七八分酒意,狂兴不禁,乃对月寓怀,口号一绝云:

\begin{verse} \wei
    时逢三五便团圆,满把晴光护玉栏。\\
    天上一轮才捧出,人间万姓仰头看。
\end{verse}

    士隐听了,大叫:``妙哉!吾每谓兄必非久居人下者,今所吟之句,飞腾之兆已见,
不日可接履于云霓之上矣。可贺,可贺!''乃亲斟一斗为贺。雨村因干过,叹道:``非晚生酒
后狂言,若论时尚之学,晚生也或可去充数沽名,只是目今行囊路费一概无措,神京路远,
非赖卖字撰文即能到者。''士隐不待说完,便道:``兄何不早言。愚每有此心,但每遇兄时,
兄并未谈及,愚故未敢唐突。今既及此,愚虽不才,`义利'二字却还识得。且喜明岁正当
大比,兄宜作速入都,春闱一战,方不负兄之所学也。其盘费余事,弟自代为处置,亦不枉
兄之谬识矣!''当下即命小童进去,速封五十两白银,并两套冬衣。又云:``十九日乃黄道
之期,兄可即买舟西上,待雄飞高举,明冬再晤,岂非大快之事耶!''雨村收了银衣,不过
略谢一语,并不介意,仍是吃酒谈笑。那天已交了三更,二人方散。士隐送雨村去后,回房
一觉,直至红日三竿方醒。因思昨夜之事,意欲再写两封荐书与雨村带至神都,使雨村投
谒个仕宦之家为寄足之地。 因使人过去请时,那家人去了回来说:``和尚说,贾爷今日五
鼓已进京去了,也曾留下话与和尚转达老爷,说`读书人不在黄道黑道,总以事理为要,
不及面辞了。'''士隐听了,也只得罢了。真是闲处光阴易过,倏忽又是元霄佳节矣。士隐
命家人霍启抱了英莲去看社火花灯,半夜中,霍启因要小解,便将英莲放在一家门槛上
坐着。待他小解完了来抱时,那有英莲的踪影?急得霍启直寻了半夜,至天明不见,那霍
启也就不敢回来见主人,便逃往他乡去了。那士隐夫妇,见女儿一夜不归,便知有些不妥,
再使几人去寻找,回来皆云连音响皆无。夫妻二人,半世只生此女,一旦失落,岂不思想,
因此昼夜啼哭,几乎不曾寻死。看看的一月,士隐先就得了一病,当时封氏孺人也因思女
构疾,日日请医疗治。

    不想这日三月十五,葫芦庙中炸供,那些和尚不加小心,致使油锅火逸,便烧着窗
纸。 此方人家多用竹篱木壁者,大抵也因劫数,于是接二连三,牵五挂四,将一条街烧
得如火焰山一般。 彼时虽有军民来救,那火已成了势,如何救得下?直烧了一夜,方渐
渐的熄去,也不知烧了几家。只可怜甄家在隔壁,早已烧成一片瓦砾场了。只有他夫妇
并几个家人的性命不曾伤了。 急得士隐惟跌足长叹而已。只得与妻子商议,且到田庄
上去安身。偏值近年水旱不收,鼠盗蜂起,无非抢田夺地,鼠窃狗偷,民不安生,因此官
兵剿捕,难以安身。士隐只得将田庄都折变了,便携了妻子与两个丫鬟投他岳丈家去。

    他岳丈名唤封肃,本贯大如州人氏,虽是务农,家中都还殷实。今见女婿这等狼狈
而来,心中便有些不乐。幸而士隐还有折变田地的银子未曾用完,拿出来托他随分就
价薄置些须房地,为后日衣食之计。那封肃便半哄半赚,些须与他些薄田朽屋。士隐乃
读书之人,不惯生理稼穑等事,勉强支持了一二年,越觉穷了下去。封肃每见面时,便
说些现成话,且人前人后又怨他们不善过活,只一味好吃懒作等语。士隐知投人不着,
心中未免悔恨,再兼上年惊唬,急忿怨痛,已有积伤,暮年之人,贫病交攻,竟渐渐的露
出那下世的光景来。

    可巧这日拄了拐杖挣挫到街前散散心时,忽见那边来了一个跛足道人,疯癫落脱,
麻屣鹑衣,口内念着几句言词,道是:

\begin{verse} \wei
    世人都晓神仙好,惟有功名忘不了!\\
    古今将相在何方?荒冢一堆草没了。\\
    世人都晓神仙好,只有金银忘不了!\\
    终朝只恨聚无多,及到多时眼闭了。\\
    世人都晓神仙好,只有姣妻忘不了!\\
    君生日日说恩情,君死又随人去了。\\
    世人都晓神仙好,只有儿孙忘不了!\\
    痴心父母古来多,孝顺儿孙谁见了?
\end{verse}

    士隐听了,便迎上来道:``你满口说些什么?只听见些`好'`了'`好'`了'。那道人
笑道:``你若果听见`好'`了'二字,还算你明白。可知世上万般,好便是了,了便是好。
若不了,便不好,若要好,须是了。我这歌儿,便名$<<$好了歌$>>$''士隐本是有宿慧的,
一闻此言,心悟。因笑道:``且住!待我将你这$<<$好了歌$>>$解注出来何如?''道人笑道:
``你解,你解。''士隐乃说道:

\begin{verse} \wei
    陋室空堂,当年笏满床,\\
    衰草枯杨,曾为歌舞场。\\
    蛛丝儿结满雕梁,绿纱今又糊在蓬窗上。\\
    说什么脂正浓,粉正香,如何两鬓又成霜?\\
    昨日黄土陇头送白骨,今宵红灯帐底卧鸳鸯。\\
    金满箱,银满箱,展眼乞丐人皆谤。\\
    正叹他人命不长, 那知自己归来丧!\\
    训有方,保不定日后作强梁。\\
    择膏粱,谁承望流落在烟花巷!\\
    因嫌纱帽小,致使锁枷杠,\\
    昨怜破袄寒,今嫌紫蟒长:\\
    乱烘烘你方唱罢我登场,反认他乡是故乡。\\
    甚荒唐,到头来都是为他人作嫁衣裳!
\end{verse}

    那疯跛道人听了,拍掌笑道:``解得切,解得切!''士隐便说一声``走罢!''
将道人肩上褡裢抢了过来背着,竟不回家,同了疯道人飘飘而去。当下烘动街坊,众人当作
一件新闻传说。封氏闻得此信,哭来,只得与父亲商议,遣人各处访寻,那讨音信?无奈何,
少不得依靠着他父母度日。幸而身边还有两个旧日的丫鬟伏侍,主仆三人,日夜作些针线
发卖,帮着父亲用度。那封 肃虽然日日抱怨,也无可奈何了。

    这日,那甄家大丫鬟在门前买线,忽听街上喝道之声,众人都说新太爷到任。丫鬟
于是隐在门内看时, 只见军牢快手,一对一对的过去,俄而大轿抬着一个乌帽猩袍的
官府过去。 丫鬟倒发了个怔,自思这官好面善,倒象在那里见过的。于是进入房中,也
就丢过不在心上。 至晚间,正待歇息之时,忽听一片声打的门响,许多人乱嚷,说:``本
府太爷差人来传人问话。''封肃听了,唬得目瞪口呆,不知有何祸事。

\clearpage

\end{document}

